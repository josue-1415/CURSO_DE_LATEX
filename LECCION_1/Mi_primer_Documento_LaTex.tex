\documentclass{article}        %tipo de documento
\usepackage[utf8]{inputenc}    %paquete de codificación
\usepackage[spanish]{babel}    %Idioma
\usepackage{verbatim}

\begin{document}

    %Comillas simples `texto' produce ‘texto’
    %Comillas dobles ``texto'' produce “texto”.
    
    El principal grupo en ese momento correspondía al que desde los años setenta fue ``bautizado'' como grupo Suramericana, y que algunos llamaban el Sindicato Antioqueño y otros el Grupo Empresarial Antioqueño, con usos activos equivalentes al 15.7\% del PIB, unos \$ 11.500 millones de dólares estadounidenses, cuando en los años setenta ocupaba el cuarto puesto,con activos equivalentes al 7.3\% del PIB; es decir, más que duplicó su peso relativo.
    
    % Uso del simbolo $ abre el modo matemático
    
    Sean tres enteros tales que $ c=a-b+1 $. %....$ expresión matemática $....
    
    Diferencia cuando se usa y, cuando no se usa el símbolo \$ para expresiones matemáticas: 
    \begin{itemize}
        \item Sean tres enteros tales que $ c=a-b+1 $.
        \item Sean tres enteros tales que c=a-b+1 .
    \end{itemize}
    
    %Escritura de una ecuación utilizando superíndices y subíndices.
    
    % Superíndices usar "^"
    \textbf{Superíndices:}
    \begin{itemize}
        % base^superíndice_simple.
        \item Superíndices Simples:Teorema de pitágoras, $ c^2=a^2+b^2 $.          
        % base^{superíndice_compuesto}
        \item Superíndices compuestos: Ecuación exponencial $ e^{2x+1}=e$.
    \end{itemize}
    
    % Subíndices usar "_" (guión bajo)
    \textbf{Subíndices}
    \begin{itemize}
        % base_subíndice.simple
        \item Subíndices simples: El último término es $ b^n=C_nb^{n^2} $.
        % base_{subíndice.compuesto}
        \item Subíndices Compuesto: $ (a+b)^n=_nC_{r-1}a^{n-r-1}b^{r-1} $.
    \end{itemize}
       
       
     Un polinomio tiene la forma 
     $$ P(x)= a_nx^n + a_{n-1}x^{n-1} + a_{n-2}x^{n-2} + ... + a_1x^1 + a_0 $$
    {\bf Listas:}
    \begin{itemize}
        \item Josué
        \item Daniel
        \item Isaula 
        \item Avila
    \end{itemize}

    \begin{verbatim}
        \begin{document}
        
        \end{document}            
    \end{verbatim}        
    
    \newpage %nueva página..
    
    {\Huge El uso de Ambientes en \LaTeX{}} 
    
    
    
    
    
    
    
    
    
    
    
    
    
    
    
    
    
    
    
    
    
    
    
    
\end{document}